% Options for packages loaded elsewhere
\PassOptionsToPackage{unicode}{hyperref}
\PassOptionsToPackage{hyphens}{url}
\documentclass[
]{article}
\usepackage{xcolor}
\usepackage[margin=1in]{geometry}
\usepackage{amsmath,amssymb}
\setcounter{secnumdepth}{5}
\usepackage{iftex}
\ifPDFTeX
  \usepackage[T1]{fontenc}
  \usepackage[utf8]{inputenc}
  \usepackage{textcomp} % provide euro and other symbols
\else % if luatex or xetex
  \usepackage{unicode-math} % this also loads fontspec
  \defaultfontfeatures{Scale=MatchLowercase}
  \defaultfontfeatures[\rmfamily]{Ligatures=TeX,Scale=1}
\fi
\usepackage{lmodern}
\ifPDFTeX\else
  % xetex/luatex font selection
\fi
% Use upquote if available, for straight quotes in verbatim environments
\IfFileExists{upquote.sty}{\usepackage{upquote}}{}
\IfFileExists{microtype.sty}{% use microtype if available
  \usepackage[]{microtype}
  \UseMicrotypeSet[protrusion]{basicmath} % disable protrusion for tt fonts
}{}
\makeatletter
\@ifundefined{KOMAClassName}{% if non-KOMA class
  \IfFileExists{parskip.sty}{%
    \usepackage{parskip}
  }{% else
    \setlength{\parindent}{0pt}
    \setlength{\parskip}{6pt plus 2pt minus 1pt}}
}{% if KOMA class
  \KOMAoptions{parskip=half}}
\makeatother
\usepackage{longtable,booktabs,array}
\usepackage{calc} % for calculating minipage widths
% Correct order of tables after \paragraph or \subparagraph
\usepackage{etoolbox}
\makeatletter
\patchcmd\longtable{\par}{\if@noskipsec\mbox{}\fi\par}{}{}
\makeatother
% Allow footnotes in longtable head/foot
\IfFileExists{footnotehyper.sty}{\usepackage{footnotehyper}}{\usepackage{footnote}}
\makesavenoteenv{longtable}
\usepackage{graphicx}
\makeatletter
\newsavebox\pandoc@box
\newcommand*\pandocbounded[1]{% scales image to fit in text height/width
  \sbox\pandoc@box{#1}%
  \Gscale@div\@tempa{\textheight}{\dimexpr\ht\pandoc@box+\dp\pandoc@box\relax}%
  \Gscale@div\@tempb{\linewidth}{\wd\pandoc@box}%
  \ifdim\@tempb\p@<\@tempa\p@\let\@tempa\@tempb\fi% select the smaller of both
  \ifdim\@tempa\p@<\p@\scalebox{\@tempa}{\usebox\pandoc@box}%
  \else\usebox{\pandoc@box}%
  \fi%
}
% Set default figure placement to htbp
\def\fps@figure{htbp}
\makeatother
\setlength{\emergencystretch}{3em} % prevent overfull lines
\providecommand{\tightlist}{%
  \setlength{\itemsep}{0pt}\setlength{\parskip}{0pt}}
\usepackage[]{natbib}
\bibliographystyle{plainnat}
\usepackage{booktabs}
\usepackage{bookmark}
\IfFileExists{xurl.sty}{\usepackage{xurl}}{} % add URL line breaks if available
\urlstyle{same}
\hypersetup{
  pdftitle={Statistical Analysis of Data in the Biotech Industry},
  pdfauthor={Francois Collin},
  hidelinks,
  pdfcreator={LaTeX via pandoc}}

\title{Statistical Analysis of Data in the Biotech Industry}
\author{\href{https://www.linkedin.com/in/francoisz/}{Francois Collin}}
\date{November 04, 2025}

\begin{document}
\maketitle

{
\setcounter{tocdepth}{2}
\tableofcontents
}
\section*{Preliminaries - Leo Breiman}\label{preliminaries---leo-breiman}
\addcontentsline{toc}{section}{Preliminaries - Leo Breiman}

Before machine learning became a household name for engineers and computer scientists
interested in making predictions, before AI had any developed applications, and
before the name ``Data Science'' was popularized to describe an area of statistical application
which most statisticians ignored, Leo Breiman \citep{Breiman:1984aa, Breiman:2001aa} was calling
for the field of statistics to modernize or become irrelevant.

\href{http://www.jstor.org/stable/2676681}{Breiman, L. (2001). Statistical modeling: The two cultures. Statistical Science 16, 199--215}
is a must-read.

My recent experience working in the context of specifying and validating an omic pipeline
used to produce a molecular diagnostic test made it clear that even without AI, DS, and ML,
statistical methods needed modernizing: the usual approach of using a named
statistical procedure off the shelf will not yield accurate results in most cases -
the esoteric error distributions which characterize modern datasets must be accounted for
when analyzing these data.
See \href{https://breimansworld.netlify.app/}{Where's the Randomness?} for a
description of this context.\footnote{
  Why the use of canned procedures persists is easily explained
  in the context of a client who is uncritical of the form of any proposed solution
  while being unflinchingly insistent on the timely
  release of the said solution.}

Inspired and encouraged by Leo Breiman, Bin Yu has modernized the concept of variability in data
to reflect the character of modern day data collection and analysis. The PQRS framework
(Yu and Kumbier, Front Inform Technol Electron Eng 2018 19(1):6-9)) highlight 4 elements of
statistical practice which remain crucially important for the successful analysis of modern
day datasets: population (P), question of interest (Q), representativeness of training data (R),
and scrutiny of results (S). These concepts have always been important, but very often
neglected in practice.. These concepts have always been important, but very often
neglected in practice.

\section{Statistics Training}\label{statistics-training}

\subsection{\texorpdfstring{Training Program for Statisticians - a collaboration with \textbf{claude.ai}}{Training Program for Statisticians - a collaboration with claude.ai}}\label{training-program-for-statisticians---a-collaboration-with-claude.ai}

As an exercise working with claude.ai, we design a training program for experienced statisticians.
The professional development training is meant to bring statisticians up to speed with
novel statistical methods and techniques which may be outsite of their expertise or
sector of industry.

\begin{itemize}
\item
  \href{./_M1C-stats-pd-plan-final-v35.html}{stats-pd-plan-final-v35.html} - final version of the
  professional development program.
\item
  \href{./_M1C-stats-pd-plan-change-log.html}{stats-pd-plan-change-log.html} -
  detailed change log documenting the development process
\item
  \href{./_M1C-ai-assistant-guide.html}{ai-assistant-guide} - outline
  principles which apply broadly to working with AI assistants on complex projects.
\end{itemize}

\subsection{Specific Training Resources}\label{specific-training-resources}

\subsubsection*{Generative AI}\label{generative-ai}
\addcontentsline{toc}{subsubsection}{Generative AI}

Resources - online courses, published papers, books, slide presentations - to learn
about the areas of applications and development of AI that are most important to statisticians.
These should include a basic set of learnings which all professionals interested in
harnessing the powers of AI should know. In addition to this basic set, learnings
that are most relevant for the integration of AI with statistics.

Yu and Kumbier (Front Inform Technol Electron Eng 2018 19(1):6-9) \citep{Yu:2018aa}
provide a framework for integrating statistical ideas in AI work which they
term the PQRS Workflow. In this framework, the statistical concepts of
population (P), question of interest (Q), representativeness of training data (R),
and scrutiny of results (S) remain critically important in the application
of AI to data analysis. We should include any training which can help clarify the framework
to non-statisticians, or help statisticians explain the framework to
non-statisticians.

If it helps to limit resources or choose among similar options, we can assume
that the statisticians work with bio-tech companies which market molecular diagnostics
devices which use omic-wide\footnote{
  transcriptomic, genomic, proteomic, methylomic}
profiels as input.

\begin{itemize}
\tightlist
\item
  Report: \href{./_M1D-genAI-report-updated.html}{Comprehensive Introduction to Generative AI Resources}
\end{itemize}

\begin{itemize}
\tightlist
\item
  Resources:

  \begin{itemize}
  \tightlist
  \item
    \href{./_M1D-genAI-changelog.html}{Gen AI - ChangeLog}
  \item
    \href{./_M1D-genAI-instructions.html}{Gen AI - Instructions}
  \end{itemize}
\end{itemize}

\subsubsection*{Deep Learning Models}\label{deep-learning-models}
\addcontentsline{toc}{subsubsection}{Deep Learning Models}

Resources - online courses, published papers, books, slide presentations - to learn
about the development deep learning theiry ands applications
that are most important to statisticians.
These should include a basic set of learnings which all professionals interested in
harnessing the powers of Deep Leaning should know. In addition to this basic set, learnings
that are most relevant for the integration of Deep Learning with statistics.

If it helps to limit resources or choose among similar options, we can assume
that the statisticians work with bio-tech companies which market molecular diagnostics
devices which use omic-wide\footnote{
  transcriptomic, genomic, proteomic, methylomic}
profiels as input.

\begin{itemize}
\tightlist
\item
  Report: \href{./_M1E-DL-deep-learning-resources.html}{Comprehensive Resources for Deep Learning in Statistics \& Biotech}
\end{itemize}

\begin{itemize}
\tightlist
\item
  Resources:

  \begin{itemize}
  \tightlist
  \item
    \href{./_M1E-DL-changelog.html}{ChangeLog}
  \item
    \href{./_M1E-DL-repository-setup.html}{Repository Setup}
  \item
    \href{./_M1E-DL-rendering-instructions.html}{Instructions for Rendering Reports and Changelog}
  \item
    \href{./_M1E-DL-report-instructions.html}{Generating HTML Reports from Markdown}
  \item
    \href{./_M1E-DL-markdown-template.html}{Markdown Template}
  \end{itemize}
\end{itemize}

\section{Data (Re-)Analysis by Topic}\label{data-re-analysis-by-topic}

\subsection{Reference-Free Genomic Inference with SPLASH and OASIS}\label{reference-free-genomic-inference-with-splash-and-oasis}

\begin{itemize}
\tightlist
\item
  \href{_M3B-SPLASH-OASIS/splash_oasis_methodology_presentation_v2.html}{Presentation}

  \begin{itemize}
  \tightlist
  \item
    Summarizes the key points made in the SPLASH and OASIS papers
  \item
    Includes comments on alternative approaches or interpretations
    \textbf{but no additional analyses}
  \end{itemize}
\end{itemize}

\begin{itemize}
\tightlist
\item
  \href{./_M2B-RFGI_SPLASH_v2.html}{SPLASH} - a detailed look at the SPLASH article.
\end{itemize}

\begin{itemize}
\tightlist
\item
  \href{./_M3A-OASIS.html}{OASIS} - a detailed look at the OASIS article.

  \begin{itemize}
  \tightlist
  \item
    INCOMPLETE
  \end{itemize}
\end{itemize}

\subsection{scRNAseq UMI count normalization}\label{scrnaseq-umi-count-normalization}

\begin{itemize}
\item
  slide deck: OneDrive folder of Speed Lab Meeting: \url{https://outlook.office365.com/mail/group/wehi.edu.au/speedlabmeeting/files}
\item
  Reference:

  \begin{itemize}
  \tightlist
  \item
    Ahlmann-Eltze and Huber (2023) \citep{Ahlmann-Eltze:2023aa}
  \end{itemize}
\item
  Other:

  \begin{itemize}
  \tightlist
  \item
    Melms et. al.~(2021) \citep{Melms:2021aa}
  \end{itemize}
\end{itemize}

\subsection{RUV}\label{ruv}

\subsubsection*{Normalization Lit Review}\label{normalization-lit-review}
\addcontentsline{toc}{subsubsection}{Normalization Lit Review}

\begin{itemize}
\tightlist
\item
  To Be Completed with an appropriate assistant.
\end{itemize}

\subsubsection*{The latest RUV}\label{the-latest-ruv}
\addcontentsline{toc}{subsubsection}{The latest RUV}

\begin{itemize}
\tightlist
\item
  Link to Ramyar Molania presentations.
\end{itemize}

\subsubsection*{Beyond RUV}\label{beyond-ruv}
\addcontentsline{toc}{subsubsection}{Beyond RUV}

Can the methodology which enables the basic RUV analysis be adapted to
other applications?

\begin{itemize}
\tightlist
\item
  IUV - identification of unwanted variability

  \begin{itemize}
  \tightlist
  \item
    the automated high throughput processing of samples necessary
    to get the omic based readouts used to diagnose samples is carried out by
    a pipeline which strings together numerous steps (typically hundreds).

    \begin{itemize}
    \tightlist
    \item
      eg. Cell-Free DNA Methylation Profiling Analysis
    \end{itemize}
  \item
    As samples proceed through the pipeline they will be organized in various configurations
    giving rise to various groupings:

    \begin{itemize}
    \tightlist
    \item
      plasma isolation
    \item
      cfDNA extraction
    \item
      amplification, conversion, \ldots{}
    \end{itemize}
  \item
    measurements are recorded as samples travel through the pipeline, each
    measurement potentially having its own shared variability or dependency structure
  \item
    a real problem is to identify the factors giving rise to excessive variability
    in the downstream read-outs
  \end{itemize}
\end{itemize}

\begin{itemize}
\tightlist
\item
  QUV - quantification of unwanted variability

  \begin{itemize}
  \tightlist
  \item
    can RUV methodology be adapted to provide quantification of reproducibility?
  \end{itemize}
\end{itemize}

\subsection{Miscellanious}\label{miscellanious}

\begin{itemize}
\tightlist
\item
  \href{./_M4A-intro_IUV.html}{IUV}
\item
  \href{./_M5A-cf_ddPCR_seq.html}{Digital Driplet PCR - soon}
\item
  \href{./_M6A-randomness_in_calls.html}{Randomness in sample calls - soon}
\item
  \href{./_M7A-interimSampleSizeAnalysis.html}{interimSampleSizeAnalysis - soon}

  \begin{itemize}
  \tightlist
  \item
    {[}\_M3A\_study\_design\_CB\_notes.pdf{]}
  \end{itemize}
\item
  \url{nanopore_talk/microbiome_slides.html}

  \begin{itemize}
  \tightlist
  \item
    \url{nanopore_talk/nanopore_background.html}
  \item
    \url{nanopore_talk/umap_explained.html}
  \item
    \url{nanopore_talk/pdf_image_extraction.html}
  \end{itemize}
\end{itemize}

\section{References}\label{references}

\section{Appendix:}\label{appendix}

\subsection*{Challenges in Statistical Analysis}\label{challenges-in-statistical-analysis}
\addcontentsline{toc}{subsection}{Challenges in Statistical Analysis}

`

Types of errors:

\begin{itemize}
\tightlist
\item
  The question is wrong or inadequately posed

  \begin{itemize}
  \tightlist
  \item
    this is not that infrequent and is largely caused by
    folks phrasing the question in a form that anticipates the solution.
    eg, client expresses desire to know what a particular estimated regression
    coefficient is
  \item
    solution is easy: subject matter experts should think hard about the
    question which is then phrased using appropriate subject matter specific language.
  \end{itemize}
\end{itemize}

\begin{itemize}
\tightlist
\item
  An inappropriate model is used as the analysis framework to
  address a question

  \begin{itemize}
  \tightlist
  \item
    this is either due to common practice or inadequate statistical training
  \item
    David Freedman discussed the inappropriate use of
    statistical models in the social sciences
    \citep{Freedman:2008aa, Freedman:2008ad, Freedman:2008ac, Freedman:2009ac}
  \item
    Inappropriate modeling is not limited to the social sciences

    \begin{itemize}
    \tightlist
    \item
      in biotech, the unchecked use of logistic regression whenever the
      response is binary is ubiquitous.

      \begin{itemize}
      \tightlist
      \item
        the results of such analyses could be misleading due to the biases
        illustrated in Freedman (2008) \citep{Freedman:2008aa}
      \end{itemize}
    \end{itemize}
  \item
    Leo Breiman pointed to inadequate training as the cause of
    the reliance on \textbf{standard textbook methods} which
    have a limited range of applicability - Breiman:
    \citep{Breiman:1984aa, Breiman:2001aa}
  \item
    as regression models are used to answer almost every question
    that come up in the internal analyses conducted by companies in industry,
    many are going to be flawed.

    \begin{itemize}
    \tightlist
    \item
      For a review of how regression models go wrong see
      \citep{Freedman:2009aa, Freedman:2010aa, Freedman:2008ac, Freedman:2008ad, Freedman:2004aa, Freedman:2008aa, Freedman:2008ab}
    \end{itemize}
  \item
    clinical validation studies for complex omic Dx instruments that are designed
    according to FDA guidelines will make use of the classical Neyman-Pearson hypothesis
    testing context which is often invalid:

    \begin{itemize}
    \tightlist
    \item
      the samples in the study may not be representative of the target population.

      \begin{itemize}
      \tightlist
      \item
        there is rarely any explicit randomness in the selection of the study samples
        so inference to any superset of the study samples is iffy at best.
      \end{itemize}
    \item
      dependency structure among samples processed in batches is often complex

      \begin{itemize}
      \tightlist
      \item
        a run may be treated as a batch with internal structure or sub-batches.
      \end{itemize}
    \item
      N is often small
    \item
      contrived samples may not be interpretable as intended or required
    \item
      other???
    \end{itemize}
  \end{itemize}
\end{itemize}

Improving in Analysis Results:

\begin{itemize}
\tightlist
\item
  Assessment, Assessment, Assessment

  \begin{itemize}
  \tightlist
  \item
    the path to better results through improved methodology must go through
    \textbf{valid assessments of performance} - Speed:
    \citep{Irizarry:2003aa, Bolstad:2004aa, Gagnon-Bartsch:2012aa, Gandolfo:2018aa, Irizarry:2006aa, Jacob:2018aa, Risso:2014aa, Yuan:2019aa}
  \item
    \emph{valid} here means that the assessment provides a direct indicator of performance
    on a meaningful scale or end point.

    \begin{itemize}
    \tightlist
    \item
      when normalizing RNASeq gene expression profiles, one can measure the effectiveness of
      the normalization step by examining the similarities among gene expression profiles before
      and after normalization. Although this is a sensible way to assess the effect of the
      normalization step, it is not a good metric to assess the impact normalization has on
      the ultimate goal of the analysis - DGE analysis.
    \item
      when comparing two sequencing instruments in terms of their ability to carry-out
      a reagent QC protocol designed to assess reagent quality:

      \begin{itemize}
      \tightlist
      \item
        although the appropriate assessment probably calls for an assessment of
        performance against the truth, and a comparison of these performance parameters between
        instruments, regulatory compliance may call for an equivalency assessment

        \begin{itemize}
        \tightlist
        \item
          it should not take much reflecting realize that an equivalency assessment
          creates unnecessary complications and can go wrong in many ways:

          \begin{itemize}
          \tightlist
          \item
            an obvious problem being drift.
          \item
            down sampling to make the two instruments comparable is another obvious
            problem
          \end{itemize}
        \end{itemize}
      \end{itemize}
    \item
      using an upstream measurement to predict unconverted fraction
      -SCALE, SCALE, SCALE
    \end{itemize}
  \end{itemize}
\end{itemize}

\section{Bad Analysis by Example}\label{bad-analysis-by-example}

\begin{itemize}
\item
  predict \%unconverted with assistant tech
\item
  DDpcr
\end{itemize}

\section{}\label{section}

\_M0A-healthy\_aging\_genes.html

\_M1A-prof\_dev\_plan\_v1.html

\_M1B-prof\_dev\_plan.html

\_M1C-ai-assistant-guide.html

\_M1C-stats-pd-plan-change-log.html

\_M1C-stats-pd-plan-final-v35\_old.html

\_M1C-stats-pd-plan-final-v35.html

\_M1D-genAI-changelog.html

\_M1D-genAI-instructions.html

\_M1D-genAI-report-html.html

\_M1D-genAI-report-updated.html

\_M1E-DL-changelog.html

\_M1E-DL-deep-learning-resources.html

\_M1E-DL-markdown-template.html

\_M1E-DL-rendering-instructions.html

\_M1E-DL-report-instructions.html

\_M1E-DL-repository-setup.html

\_M2A-Comments.html

\_M2A-RFGI\_SPLASH\_v1.html

\_M2B-Comments.html

\_M2B-RFGI\_SPLASH\_v2.html

\_M3A-OASIS.html

\_M3B-SPLASH-OASIS-changelog.html

\_M3B-SPLASH-OASIS-markdown\_source.html

\_M3B-SPLASH-OASIS-presentation\_slides.html

\_M3B-SPLASH-OASIS-splash\_oasis\_presentation.html

\_prof\_dev\_plan.html

\_splash\_oasis\_presentation.html

index\_1.html

\bibliography{../\_bibFiles/\_healthy\_aging.bib,../../\_bibFiles/\_Breiman.bib,../../\_bibFiles/\_Freedman.bib,../../\_bibFiles/\_Yu.bib,../../\_bibFiles/\_RUV.bib,../../\_bibFiles/\_RMA.bib,../../\_bibFiles/\_scRNAseq\_norm.bib}

\end{document}
